\documentclass[a4paper]{article}

\usepackage[T1]{fontenc}
\usepackage{textcomp}
\usepackage[italian]{babel}
\usepackage{hyperref}
\usepackage{amsmath, amssymb, amsthm}
% for \lightning
\usepackage{stmaryrd}
\usepackage{geometry}
\usepackage{tikz-cd}
\usepackage{enumitem}

\hypersetup{
	colorlinks = true, % links instead of boxes
	urlcolor   = blue, % external hyperlinks
	linkcolor  = blue, % internal links
	citecolor  = red   % citations
}

\newcommand{\R}{\mathbb{R}}
\newcommand{\C}{\mathbb{C}}
\newcommand{\Q}{\mathbb{Q}}
\newcommand{\K}{\mathbb{K}}
\newcommand{\N}{\mathbb{N}}
\newcommand{\A}{\mathbb{A}}
\newcommand{\Z}{\mathbb{Z}}

\newcommand{\ssfrac}[2]{
    \raisebox{+0.3ex}{$#1$}
    /
    \raisebox{-0.3ex}{$#2$}
}

\newcommand{\sfrac}[2]{
    \raisebox{+0.3ex}{\scalebox{0.9}{$#1$}}
    /
    \raisebox{-0.3ex}{\scalebox{0.9}{$#2$}}
}

\renewcommand{\labelitemii}{$\circ$}
\renewcommand{\Im}{\operatorname{Im}}

\newcommand\numberthis{\addtocounter{equation}{1}\tag{\theequation}}

\newtheorem{theorem}{Theorem}[section]
\newtheorem{lemma}{Lemma}[section]

\theoremstyle{definition}
\newtheorem{definition}{Definition}[section]

\theoremstyle{definition}
\newtheorem{example}{Example}[section]

\theoremstyle{remark}
\newtheorem*{remark}{Remark}

\theoremstyle{definition}
\newtheorem{exercise}{Exercise}[section]

\title{Istituzioni di Algebra 2022/2023}
\author{Francesco Minnocci}
\begin{document}
\maketitle
\section*{Homework 1}
\setcounter{section}{1}
\begin{exercise}
	Consider the chain of prime ideals
	\begin{equation}\label{eq:chain}
		(0)\subsetneq\left(x  \right)\subsetneq (x,y)
	\end{equation}
	in  $\K\left[ x,y \right] $, which is maximal since we know that $\operatorname{dim}\left(
	\K[x,y] \right) =2$.\\
	Now, if we let $\pi$ be the projection homomorphism onto $$A:=\ssfrac{\K[x,y]}{(xy)},$$ we realize that $(\overline{x})$ is a prime ideal by the correspondence between prime ideals in
	the quotient, as $(x)$ contains
$\operatorname{Ker}\left( \pi \right) =(xy)$. Thus, we just need to show that there are no prime ideals below $(\overline{x})$ in $A$: $(\overline{0})$ is not prime, as $A$ has
zero-divisors (e.g. $\overline{x}$), and if there were a prime ideal $\overline{0}\subsetneq \overline{\mathfrak{p}}\subsetneq\overline{x}$ we would get a prime $\mathfrak{p}$ in between $(0)$
and $(x)$ in \eqref{eq:chain} (again by correspondence of prime ideals), which would go against $\operatorname{dim}\left( \K[x,y] \right) =2$.\qed
\end{exercise}
\begin{exercise}
	We first prove the case where $A$ is a two-dimensional local integral domain: let $\mathfrak{m}$ be the unique maximal ideal, then we get
	$\operatorname{ht}(\mathfrak{m})=\operatorname{dim}(A)=2$, and
	$(0)$ is the unique minimal prime ideal since $A$ is a domain. Therefore, it is enough to prove that $A$ has infinitely many prime ideals of height 1: if by
	contradiction there were finitely many prime ideals $$\mathfrak{p}_1,\dots,\mathfrak{p}_n$$
	strictly contained between $(0)$ and $\mathfrak{m}$, then their union cannot contain  $\mathfrak{m}$ by prime avoidance, since if it did then the maximal ideal would be
	contained in $\mathfrak{p}_k$ for some $k\in\left\{ 1,\dots,n \right\} $, which would be a contradiction. So, we can choose some element
	$$x\in\mathfrak{m}\setminus\bigcup_{k=1}^n{\mathfrak{p}_k},$$ and by construction there can not be any other prime ideal containing $x$ below $\mathfrak{m}$, which is thus
	minimal above it, which yields a contradiction by the Hauptidelsatz's bound $$\operatorname{ht}{\mathfrak{m}}\leq 1.$$
	We now move on to the general case, which we shall prove by induction: if $\operatorname{dim}{(A)}=2$ and $A$ is a Noetherian ring, then we can reduce ourselves to the
	"easier" case we just proved: consider a maximal chain which realizes the dimension $$ \mathfrak{p}\subsetneq\mathfrak{q}\subsetneq\mathfrak{m}\subsetneq A.$$ If we
	quotient out by $\mathfrak{p}$ and localize by $\mathfrak{m}$, we get a local integral domain of dimension 2, as $\mathfrak{p}$ became the prime ideal $(\bar{0})$ in the
	quotient, and localizing by $\mathfrak{m}$ didn't change the height of $\bar{\mathfrak{q}}$ (and it's still a domain after the localization, as localization and quotient commute by known facts of Algebra 2).
	Thus, by the initial case we find infinitely many prime ideals of height 1, then use localization and quotient correspondence to get infinitely many prime ideals in $A$,
	and since these are strictly contained between $\mathfrak{p}$ and $\mathfrak{m}$ they must have height $1$ by maximality of any such chain of length 2 in a two-dimensional
	ring.\\
	Now, suppose $\operatorname{dim}(A)=r+1$ and consider a maximal chain of prime ideals $$ \mathfrak{p}_0\subsetneq\mathfrak{p}_1\subsetneq\dots\subsetneq\mathfrak{p}_{r+1}
	.$$ Regarding
	heights $i\in\{1,\dots, r-1\}$ we can localize by $\mathfrak{p}_r$ and use the inductive hypothesis on $A_{\mathfrak{p}_r}$, which gives us infinitely
	many prime ideals of height $i$ which correspond bijectively to prime ideals in $A$ of the same height. Furthermore, if we localize $A$ by the maximal ideal $\mathfrak{p}_{r+1}$ of the above
	chain and then quotient it by the localization of $\mathfrak{p}_{r-1}$, we get
	$$
	\ssfrac{A_{\mathfrak{p}_{r+1}}}{\mathfrak{p}_{r-1}A_{\mathfrak{p}_{r+1}}},\cong
	(\sfrac{A}{\mathfrak{p}_{r-1}})_{\mathfrak{p}_{r+1}}
	$$
	which
	is a local, two-dimensional (the only primes of the maximal chain which survive are those which were not below $\mathfrak{p}_{r-1}$) integral domain (as we quotiented by a
	prime ideal), so we can use the
	initial case to get infinitely many prime ideals of height 1 in this ring. However, these correspond bijectively to prime ideals of height $r$ in $A$, by lifting through
	localization and quotient.\\
	Finally, in the case where $A$ is infinite-dimensional, given any integer we can pick a prime chain which is strictly longer, and localizing with respect to the highest
	prime ideal of such chain lets us get back to the situation where $A$ is finite-dimensional, use the above proof and then conclude by correspondence in the
	localization.\qed
\end{exercise}
\begin{exercise}\


	\begin{itemize}
	\item[(a)] We show both inequalities:
		\begin{itemize}
			\item $\operatorname{dim}(A)\leq\operatorname{dim}(B)$: this follows from the \textbf{Going up} Theorem, as we are in an integral extension and if
				$$(0)\subsetneq\mathfrak{p}_1\subsetneq\dots\subsetneq\mathfrak{p}_r$$ is a prime chain in $A$, by
				the theorem there exists a prime chain $$(0)\subsetneq\mathfrak{q}_1\subsetneq\dots\subsetneq\mathfrak{q}_r$$ in $B$, where $\forall
				i\in\{1,\dots,n\}\:\mathfrak{q}_i$ lies over
				$\mathfrak{p}_i$.
			\item $\operatorname{dim}(A)\geq\operatorname{dim}(B)$ is a consequence of a remark made in class, namely that in an integral extension any two prime ideals of $B$
				$$\mathfrak{q}_1\subsetneq\mathfrak{q}_2$$ cannot have the same contraction in $A$, so any prime chain in $B$ contracts to a prime
				chain of equal
				length in $A$ (as prime ideals contract to prime ideals).
		\end{itemize}
	\item[(b)] The inequality $\operatorname{ht}(\mathfrak{p})\geq\operatorname{ht}(\mathfrak{q})$ is again an application of the aforementioned remark to any prime chain in $B$ ending in
		$\mathfrak{q}$ which realizes
		$\operatorname{ht}(\mathfrak{q})$. For the other direction, let $$(0)\subsetneq\mathfrak{p}_1\subsetneq\dots\subsetneq\mathfrak{p}_r\subsetneq\mathfrak{p}$$ be a
		prime chain in $A$ which realizes $\operatorname{ht}\left( \mathfrak{p} \right) $. Note that we are in the hypothesis of the \textbf{Going Down} Theorem, so that we can
		construct the following prime chain in $B$: starting at $\mathfrak{q}=\mathfrak{p}\cap A$, where the Theorem gives us a $\mathfrak{q}_r\subsetneq\mathfrak{q}$ which
		contracts to $\mathfrak{p}_r$, we inductively \textit{go down} using the Theorem, obtaining
		$$(0)\subsetneq\mathfrak{q}_1\subsetneq\dots\subsetneq\mathfrak{q}_r\subsetneq\mathfrak{q},$$ where every $\mathfrak{q}_i$ contracts to the corresponding
		$\mathfrak{p}_i$. Thus, $\operatorname{ht}{(\mathfrak{q})}\geq\operatorname{ht}{(\mathfrak{p})}$.\qed
	\end{itemize}
\end{exercise}
\begin{exercise}(Done in collaboration with Sebastián Camponovo)\


\begin{itemize}
	\item[(a)] We prove both inclusions:
	\begin{itemize}
	\item $B\supseteq\left\{ b\in \mathbb{L} \mid \exists r\geq 0:b^{p^r}\in A \right\} $

		If $b\in \mathbb{L}$ is such that $b^{p^r}\in A$ for some  $r\geq 0$, $b$ is a root of the monic polynomial $x^{p^r}-b^{p^r}$. Therefore, $b\in L$ is integral over A and by
		definition of integral closure $b\in B$.
	\item $B\subseteq\left\{ b\in \mathbb{L} \mid \exists r\geq 0:b^{p^r}\in A \right\} $

		Pick any $b\in B$. Then, $b\in \mathbb{L}$, and by definition of purely inseparable extension there exists an $r\geq 0$ such that $b^{p^r}\in \K$. However, since in
		particular $B$ is an integral extension of $A$, every such $b$ is integral over $A$, so $b$ is a root of a monic polynomial $f\in A[x]\setminus\left\{ 0 \right\}
		$. But then, $b^{p^r}$ is a root of the monic
		polynomial $f^{p^r}$ (since $f(x)^{p^r}=f(x^{p^r})$ in characteristic $p$ by the Fröbenius endomorphism), which also has coefficients in A, meaning $b^{p^r}\in\K$ is integral over the integrally closed ring $A$, and consequently $b^{p^r}\in A$.
	\end{itemize}
	\item[(b)] Let $$\nu: \K\to\Z\cup\left\{ +\infty \right\} $$ be a discrete valuation on the DVR $A$. Then, we can define the following valuation on $B$: we have showed in
		class that in a purely inseparable extension there exists an $N\gg 0$ such that $\mathbb{L}^{p^N}\in\K$, so given such $N$ we define
		\[
			\begin{array}{ccccc}
				&\eta:&\text{Frac}\left( B \right) &\longrightarrow&\Z\cup\left\{ +\infty \right\}\\
				     &&\displaystyle\frac{b}{b'}&\longmapsto&\nu{(b^{p^N})}-\nu{(b'^{p^N})}.
			\end{array}
		\]
		Now, it is enough to show that $\eta$ is a valuation on elements of $B$ (and then by construction $B$ will be the valuation ring of $\eta$, and thus a DVR):
		indeed, take any two $b,b'\in B$, then
		\begin{itemize}
			\item $\eta(b\cdot b')=\nu((b\cdot b')^{p^N})=\nu(b^{p^N})\cdot\nu(b'^{p^N})=\eta(b)+\eta(b')$, since $\nu$ is a valuation;
			\item $\eta(b+
			b')=\nu((b+b')^{p^N})=\nu(b^{p^N}+{b'}^{p^N})\leq\text{min}{\left\{\nu(b^{p^N}),\nu(b'^{p^N})\right\}}=\text{min}{\left\{\eta(b),\eta(b')\right\}}$, where the second equality follows from the Fröbenius
				endomorphism, and the inequality holds because $\nu$ is a valuation;
			\item $\eta(b)=+\infty\iff\nu(b^{p^N})=+\infty\iff b^{p^N}=0\iff b=0$ since again $\nu$ is a valuation.
		\end{itemize}
		\qed
\end{itemize}
\end{exercise}
\begin{exercise}
	We will see that $\mathfrak{q}=(y,\frac{x}{y})$ is a maximal ideal of $B$, and thus prime: indeed, since we can write $x$ as  $y\cdot \frac{x}{y}$, we have that $\mathfrak{q}=(x,y,\frac{x}{y})$, so
	taking the quotient $\sfrac{B}{\mathfrak{q}}$ we get the base field $\K$:
	$$\ssfrac{B}{\mathfrak{q}}=\frac{\K[x,y,\frac{x}{y}]}{\left(x,y,\frac{x}{y}\right)}\cong\frac{\K[x,y,z]}{\left( x,y,z,zy-x \right) }\cong \frac{\K[x,y,z]}{\left( x,y,z
	\right) }\cong\K.$$ Therefore, $\mathfrak{q}$ is maximal. Now, $\mathfrak{q}\cap A=(x,y,\frac{x}{y})\cap
	\K[x,y]\supseteq(x,y)=\mathfrak{p}$, but since $\mathfrak{p}$ is maximal we get $\mathfrak{q}\cap A=\mathfrak{p}.$

	Let's compute the height of the two ideals: we know that $\operatorname{ht}(\mathfrak{p})=\operatorname{dim}\left( A \right)=2$, and since $\mathfrak{q}$ is maximal its
	height is equal to $\operatorname{dim}(B)$. For finitely generated $\K$-algebras, we have showed that dimension and transcendence degree coincide, and
	$\text{tr.}\text{deg.}_{\K}(B)=2$ since $\left\{ x,y \right\} $ is a maximal subset of
	algebraically independent elements in $B$ (if we were to add $\frac{x}{y}$ to such set, we would get the algebraic dependency relation $x=y\cdot \frac{x}{y}).$ In conclusion,
	$\operatorname{ht}(\mathfrak{p})=\operatorname{ht}(\mathfrak{q})=2$.

	Regarding $\sfrac{B_{\mathfrak{q}}}{\mathfrak{p}B_{\mathfrak{q}}}$, we observe that
	\begin{equation}\label{eq:isomorphism}
		\ssfrac{B_{\mathfrak{q}}}{\mathfrak{p}B_{\mathfrak{q}}}\cong(\ssfrac{B}{\mathfrak{p}B})_{\overline{\mathfrak{q}}},
	\end{equation}
	and since $\mathfrak{p}B$ is the ideal generated by $\{x,y\}$ in B, we have
	$$ \ssfrac{B}{\mathfrak{p}B}=\frac{\K\left[x,y,\frac{x}{y}\right]}{(x,y)}\cong\frac{\K[x,y,z]}{(x,y,zy-x)}\cong\frac{\K[x,y,z]}{\left( x,y \right) }\cong\K\left[z\right] .$$
	Now, the ideal $\overline{\mathfrak{q}}$ is identified with $\left(z\right)$ in the above isomorphism, and as such it is the unique maximal ideal of $\sfrac{B}{\mathfrak{p}B}$. In conclusion,
\eqref{eq:isomorphism} tells us that $$ \ssfrac{B_{\mathfrak{q}}}{\mathfrak{p}B_{\mathfrak{q}}}\cong\K[z]_{(z)} \implies \operatorname{dim}(\ssfrac{B_{\mathfrak{q}}}{\mathfrak{p}B_{\mathfrak{q}}}) =
\operatorname{dim}(\K[z]_{(z)})=1,$$
which yields the desired strict inequality:
	$$ \operatorname{ht}(\mathfrak{q})<\operatorname{ht}(\mathfrak{p})+\operatorname{dim}(\ssfrac{B_{\mathfrak{q}}}{\mathfrak{p}B_{\mathfrak{q}}}) .$$\qed
\end{exercise}
\end{document}

\documentclass[a4paper]{article}

\usepackage[T1]{fontenc}
\usepackage{textcomp}
\usepackage[italian]{babel}
\usepackage{hyperref}
\usepackage{amsmath, amssymb, amsthm}
% for \lightning
\usepackage{stmaryrd}
\usepackage{geometry}
\usepackage{tikz}
\usepackage{tikz-cd}
\usepackage{enumitem}

\hypersetup{
	colorlinks = true, % links instead of boxes
	urlcolor   = blue, % external hyperlinks
	linkcolor  = blue, % internal links
	citecolor  = red   % citations
}

\newcommand{\R}{\mathbb{R}}
\newcommand{\C}{\mathbb{C}}
\newcommand{\Q}{\mathbb{Q}}
\newcommand{\K}{\mathbb{K}}
\newcommand{\N}{\mathbb{N}}
\newcommand{\A}{\mathbb{A}}
\newcommand{\Z}{\mathbb{Z}}

\newcommand{\ssfrac}[2]{
    \raisebox{+0.3ex}{$#1$}
    /
    \raisebox{-0.3ex}{$#2$}
}

\newcommand{\sfrac}[2]{
    \raisebox{+0.3ex}{\scalebox{0.9}{$#1$}}
    /
    \raisebox{-0.3ex}{\scalebox{0.9}{$#2$}}
}

\newcommand\vdashline{
  \tikz[baseline]\draw[thick, dashed](0,-\dp\strutbox)--(0,\ht\strutbox);
}

\renewcommand{\labelitemii}{$\circ$}
\renewcommand{\Im}{\operatorname{Im}}

\newcommand\numberthis{\addtocounter{equation}{1}\tag{\theequation}}

\newtheorem{theorem}{Theorem}[section]
\newtheorem{lemma}{Lemma}[section]

\theoremstyle{definition}
\newtheorem{definition}{Definition}[section]

\theoremstyle{definition}
\newtheorem{example}{Example}[section]

\theoremstyle{remark}
\newtheorem*{remark}{Remark}

\theoremstyle{definition}
\newtheorem{exercise}{Exercise}[section]

\title{Istituzioni di Algebra 2022/2023}
\author{Francesco Minnocci}
\begin{document}
\maketitle
\section*{Homework 2}
\setcounter{section}{2}
\begin{exercise}
	Since $A,B$ are Noetherian, by the \textbf{Theorem} on Dimension of Fibres we have the inequality $$ \operatorname{ht} \left( Q \right) \leq\operatorname{ht}\left( P \right) +
	\operatorname{dim}\left( \frac{B_Q}{PB_Q} \right)  .$$
	For the other inequality, we first replace $A,B$ by $A_P,B_Q$ and $P,Q$ by $P_Q,B_Q$ since by correspondance of prime ideals in localizations the heights of $P,Q$ don't
	change in the localizations. Then, set $r:=\operatorname{ht} \left( P \right)$ and $ s:=\operatorname{ht} \left( Q~\text{mod}~PB \right)$, and let
	\begin{align*}
		&{PB} \subsetneq Q_1\subsetneq Q_2\subsetneq\dots Q_s\subsetneq Q\\
		&~\vdashline\\
		P_1\subsetneq P_2\subsetneq\dots P_r\subsetneq&P
	\end{align*}
		be maximal chains of prime (except eventually $PB$) ideals realizing the heights $r,s$. Now, by the \textbf{Remark} $5.23$ of the course notes, we can use the Going
		Down property for flat extensions starting from $Q_1$ (which contracts to $P$) and inductively go down to obtain a prime chain of length $r+s$ ending in
		$Q$, which lets us conclude.
\end{exercise}
\begin{exercise}
	First notice that $(f,f')=1$ implies $(\overline{f},\overline{f}')=1$ in the residue field, as by Bezout's theorem there is a non-zero combination $$f\cdot g+f'\cdot
	h=1$$
	with  $g,h\in A[x]$, which projects to a non-zero combination in $\K[x]$ (as if $\overline{f}'$ was zero, $f$ would project to a invertible element), where  $\K$
	is the residue field of $A$. Now, $\overline{f}$ factors completely in $\K[x]$ because $\K$ is algebraically closed), and since $f$ is monic we have that
	$n=\operatorname{deg}(f)=\operatorname{deg}(\overline{f})$, so that $\overline{f}$ factors as $\prod_{i=1}^n{(x-\overline{a}_i)}$ for some $\overline{a}_i\in \K$. Moreover, we showed
that the "derivative criterion" holds for $\overline{f}$, so that the $\overline{a}_i$s must all be distinct. Thus, by the Hensel Lemma it follows that we can lift this
factorization to $$f=\prod_{i=1}^n{(x-a_i)},$$ and the ideals $(x-a_i)$ are comaximal since the ideal generated by two distinct ideals them contains the element $a_i-a_j\in A$, which is
invertible in $A$ as it projects to the non-zero element $\overline{a}_i-\overline{a}_j\in\K$.
Finally, we apply the \textbf{Chinese Remainder Theorem} to obtain $$ \ssfrac{A[x]}{(f)}=\frac{A[x]}{\prod_{i=1}^n{(x-a_i)}}\cong
\prod_{i=1}^n{\ssfrac{A[x]}{(x-a_i)}}\cong\prod_{i=1}^n{A}=A^n.$$
\end{exercise}
\begin{exercise}\

	\begin{itemize}
		\item[(a)] We claim that the unique maximal ideal of $B$ is $$\frac{(P[x],x)}{(f)},$$ where $P[x]$ indicates the extension of $P$ in $A[x]$; by correspondence, this is equivalent to show that there is an unique maximal ideal of
	$A[x]$ containing $f$.

	We claim it is enough to show that
	\begin{equation}\label{eq:claim}
		\text{Every maximal ideal of }A[x]\text{ which contains }f\text{ also contains }P[x].
	\end{equation}
	Indeed, notice that this claim concludes: with the above, any maximal ideal $M$ of $A[x]$ containing $f$ contains $(P[x],f)=(P[x],x^n)$, where $n=\operatorname{deg}{(f)}$
	and the equality follows from the
	fact that every coefficient of $f$
	excluding the leading coefficient is in $P$. With $M$ being prime, containing a power of an element means containing such element, so we have that $M$ must contain the
	maximal ideal
	$\left( P[x],x \right) $, and by maximality they are actually equal.

	We now show \eqref{eq:claim}, which by corrispondence is equivalent to showing that every maximal ideal of $B$ contains $PB$, and we do so by contradiction: suppose that $PB$ was not
	contained in some maximal ideal $M$ of $B$, then we would have $$ B=M+PB .$$ Now, since $B$ is a finitely generated module over $A$ and $P$ is equal to the Jacobson
	radical of  $A$, we can apply \textbf{Nakayama}'s Lemma (its third
	form, from the course notes for \textit{Algebra 2}) to
	obtain $M=B~\lightning$.\\
	\noindent
	Moving on to completeness, we start from the \textit{Hint}, namely that $B$ is complete with respect to the filtration ${(PB)}^k$: first notice that $$ (PB)^k=\left( P[\overline{x}] \right)
	^k=P^k[\overline{x}] ;$$ then, by the \textbf{Third isomorphism theorem}, we have
$$ \ssfrac{B}{{(PB)}^k}=\frac{\ssfrac{A[x]}{(f)}}{\ssfrac{(P^k[\overline{x}],f)}{(f)}}\cong \frac{A[x]}{\left( P^k[x],f \right) }=\frac{(\ssfrac{A}{P^k})[x]}{(f)}.$$
Now, we want to prove that $$ \varprojlim_k{\frac{\ssfrac{A}{P^k[x]}}{\left(\overline{f}^{(k)}\right)}}\cong\frac{\varprojlim_k{\left( \ssfrac{A}{P^k} \right) }[x]}{(f)}=\frac{\left(\varprojlim_k{ \ssfrac{A}{P^k} }\right)[x]}{(f)}\cong\frac{A[x]}{(f)},$$
where the last equality follows from the completeness of $A$. It therefore remains to check the first isomorphism: this is a consequence of the \textbf{Lemma} 5.10 of the course notes, since
if we consider the exact sequence
$$ 0\longrightarrow\left(\overline{f}^{(k)}\right)\longrightarrow\left(\ssfrac{A}{P^k}\right)[x]\longrightarrow\frac{\left(\ssfrac{A}{P^k}\right)[x]}{\left(\overline{f}^{(k)}\right)}\longrightarrow 0$$
then condition \textit{a)} of the aforementioned lemma holds as the maps $\left(\overline{f}^{(k+1)}\right)\rightarrow\left(\overline{f}^{(k)}\right)$ are surjective, so we have
that the induced sequence of inverse limits is exact, which concludes as $$ \varprojlim_k{\left(\overline{f}^{(k)}\right)}=\left( f \right)  .$$
To prove completeness of $B$ as a local ring, we will apply \textbf{Proposition} 5.3 of the course notes with respect to the filtrations $(PB)^k=P^k[\overline{x}]$ (with respect to which we proved
$B$ to be complete) and
$(P[\overline{x}],x)^k$ (from which would follow completeness of $B$, as it is its unique maximal ideal): on one
hand, we notice that $$ P^k[\overline{x}]\subseteq(P[\overline{x}],x)^k.$$
On the other hand, if we fix $k$ it is enough to show that, in $A[x]$, there is some $t$ such that $$((P[\overline{x}],x)^t,f)\subseteq (P^k[\overline{x}],f).$$
Indeed, since $P^j\subseteq P^j[\overline{x}] $ for any $j$ and $x^{nk}\in(P^k[\overline{x}],f)$ by definition of $f$ (with $n=\operatorname{deg}(f)$), if $t\geq k\cdot(n+1)-1$ then
$$ (P[\overline{x}],x)^t,f)=(P^t,P^{t-1}\cdot \overline{x},\dots,P^k\cdot \overline{x}^{t-k},P^{k-1}\cdot
\overline{x}^{t-k+1},\dots,\overline{x}^t,f)\subseteq (P^k[\overline{x}],f).$$
\item[(b)] Since $a_0\in P \setminus P^2$, in particular it is non-zero, so we can extend its projection to a basis of $\sfrac{P}{P^2}$. Then, any lifting of the other basis
	elements will give rise to a regular system of parameters $(a_0,c_1,\dots,c_d)$ for $A$, where $\operatorname{dim}{A}=d+1$.

	Let $Q=(P[\overline{x}],\overline{x})$ be the unique maximal ideal of $B$. Since
	\begin{equation}\label{eq:a0}
		a_0= \overline{x}^n+\dots+a_1\cdot \overline{x}
	\end{equation}
	in $B$, we claim that a regular sequence which forms a system of parameters for $B$ is given by $(\overline{x},c_1,\dots,c_d)$: indeed, $P[\overline{x}]\subseteq(\overline{x},c_1,\dots,c_d)B$ by \eqref{eq:a0}
	and obviously $\overline{x}\in(\overline{x},c_1,\dots,c_d)B$, so they are a system of generators for $Q$; the reverse inclusion
	$(\overline{x},c_1,\dots,c_d)B\subseteq\left( P[\overline{x}],x \right) $ is also clear.

	We now show that they are minimal with such property, and that they
	are a regular sequence: the former follows from the fact that we started with a regular sequence (notice that $A \text{ Noetherian}\implies B
	\text{ Noetherian}$, so \textbf{Remark} 4.8 applies and we just need to show that $\overline{x}$ is not a zero-divisor modulo $(c_1,\dots,c_d)$, but if it was then so
	would be $a_0$ by \eqref{eq:a0}, which would be a contradiction); then, to conclude we just need
	\begin{equation}\label{eq:dim}\operatorname{dim}{B}=\operatorname{dim}{A}=d+1,\end{equation} from which (by Noetherianness of $B$) \textbf{Theorem} 4.9 of
the course notes would yield the
	regularity of the aforementioned sequence of parameters (and consequently of $B$), as we would have
	identified a regular sequence of parameters which also form a minimal system of generators for the maximal ideal of $B$, as \textit{Corollary} 2.9 of the course notes says
	exactly that $Q$ cannot be generated by less than $d+1$ parameters.

	Finally, we show \eqref{eq:dim}: since $A \text{ regular local}\implies A \text{ domain}$, Eisenstein's criterion holds with the given hypothesis, and $f$ is thus irreducible in $A[x]$. As $A \text{
	regular local}\implies A \text{ UFD}$ (by the \href{https://tinyurl.com/2p87hdny}{Auslander-Buchsbaum} Theorem), $f$ is prime in
		$A[x]$, so the second exercise of the first Homework (with $A\hookrightarrow B$ being an integral extension of integral domains) tells us that
		$\operatorname{dim}{B}=\operatorname{dim}{A}$.
	\end{itemize}
\end{exercise}
\begin{exercise}\

	\begin{itemize}
		\item[(a)] By the third exercise of this Homework, the maximal ideal of $B$ is $$Q=\ssfrac{\left( (p,t)[x],x \right)}{\left( x^2+tx+p \right) }, $$ since $(p,t)$ is the maximal ideal
	of $\Z_p[[t]$. Now, $\overline{t},\;\overline{x}\in Q\implies \overline{t}\cdot\overline{x},\;\overline{x}^2\in Q^2\implies p\in Q^2$.
\item[(b)] We first prove the \textit{Hint}: this is equivalent to the claim that the nilradical of $\sfrac{C[[u]}{(p)}$ contains a prime ideal (since by Zorn's lemma it is always the
	intersection of all minimal prime ideals). Since $C$ is a DVR, let $P=(\pi)$ be its maximal ideal; then, since $p\in P$ ($C$ has residue characteristic $p$ and hence
	$\operatorname{char}(C)=0$), we have $(p)=(\pi^n)$ for some $n\geq 1$, and if we consider an element $a$ in the ideal
	$\sfrac{PC[[u]]}{(\pi^n)}$ (which is generated by $\overline{\pi}$), it must satisfy
	$$ a^n\in(\overline{\pi}^n)=0 .$$
	Now, $$\ssfrac{B}{(p)}\cong\ssfrac{\mathbb{F}_p[[t]][x]}{\left( x\cdot(x+t) \right) },$$ which is not a domain, and we consider the two ideals $(\overline{x}),
	(\overline{x}+\overline{t})$: these are prime since $$\ssfrac{\left(\sfrac{B}{(p)}\right)}{(\overline{x})}\cong\mathbb{F}_p[[t]]$$ and also
	$$\ssfrac{\left(\sfrac{B}{(p)}\right)}{(\overline{x}+\overline{t})}\cong\mathbb{F}_p[[t]],$$ which is clearly a domain.
	Since $(0)$ is not a prime ideal, these are minimal prime ideals in $\sfrac{B}{(p)}$, and they are distinct, as if a polynomial is divided by both $\overline{x}$
and $\overline{x}+\overline{t}$ then it is identically $0$.
In conclusion, they are distinct minimal prime ideals in $\sfrac{B}{(p)}$, which by $(a)$ proves that $B$ cannot be isomorphic to $C[[u]]$.
	\end{itemize}
\end{exercise}
\begin{exercise}
	Let $t$ be an uniformizer for $A$, such that $P=(t)$. Then, by \textbf{Fact} 6.15 of the course notes we have that, if $\K$ is the residue field of $A$, there is a
	Cohen ring contained in $A$, which is the ring of Witt vectors $W(\K)\subset A$ such that $P\cap W(\K)=(p)$.

	Now, as in the previous exercise $p\in P$, so $n:=\nu(p)\geq 1$. Since $A$ is a DVR and thus a domain, it is torsion-free as a $W(\K)$-module, and since $\K$ is perfect by
	\textbf{Theorem} 7.4 $W(\K)$ is also a DVR, and thus a PID. So, as suggested by the hint, we use the fact that any torsion-free module over a PID is free.

	Moreover, we show that $\left\{ 1,\overline{t},\dots,\overline{t}^{n-1} \right\}$ is a basis for the $\sfrac{W(\K)}{(t^n)}=\K$-vector space $$\ssfrac{A}{t^n A},$$ which by
	a known (\textit{Algebra 2} course notes) corollary of \textbf{Nakayama}'s Lemma will imply
that $A$ is finitely generated as a $W(\K)$-module and of rank $n$.

Indeed, take any non-zero element $b\in A$, then modulo $P=(t)$ this is equivalent to some element $w^{(1)}$ of $W(\K)$, which means that there is some $c^{(1)}\in A$
such that
$$ b=w^{(1)}+c^{(1)} t .$$
Iterating such construction, we get that $$ b=\sum_{i=1}^{n-1}{w^{(i)} t^i}+c^{(n)} t^n,$$
and modulo $t^n$ this becomes $$\overline{b}=\sum_{i=1}^{n-1}{\overline{w}^{(i)}\overline{t}^i},$$ which tells us that $\left\{ 1,\overline{t},\dots,\overline{t}^{n-1}
\right\}$ generate $\sfrac{A}{t^n A}$ as a $\K$-vector space.
We now need to show that they are linearly independent: let us take a non-trivial linear combination
$$ \sum_{i=0}^{n-1}{\overline{a}_i\overline{t}^i }=0,$$ then $a_0$ would be a combination of elements of $P=(t)$, and thus be in $W(\K)\cap P=(t^n)$ ; so $\overline{a}_0=0$ in the
quotient. Now, lifting the remaining relation yields $$ t(a_1 +\dots+ a_{n-1} t^{n-2})=d\cdot t^n $$ for some $d\in A$, and since $A$ is a domain we can divide by $t$ and then
project again to get that $\overline{a}_1\in W(\K)\cap P=(t^n)\implies \overline{a}_1=0$. Now, we can iterate the above reasoning for the rest of the coefficients to obtain the desired linear
independence.
\end{exercise}
\end{document}

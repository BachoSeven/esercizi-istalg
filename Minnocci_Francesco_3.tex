\documentclass[a4paper]{article}

\usepackage[T1]{fontenc}
\usepackage{textcomp}
\usepackage[italian]{babel}
\usepackage{hyperref}
\usepackage{amsmath, amssymb, amsthm}
% for \lightning
\usepackage{stmaryrd}
\usepackage{geometry}
\usepackage{tikz}
\usepackage{enumitem}
\usepackage{tikz-cd}

\usetikzlibrary{decorations.markings}
\tikzset{double line with arrow/.style args={#1,#2}{decorate,decoration={markings,%
mark=at position 0 with {\coordinate (ta-base-1) at (0,1pt);
\coordinate (ta-base-2) at (0,-1pt);},
mark=at position 1 with {\draw[#1] (ta-base-1) -- (0,1pt);
\draw[#2] (ta-base-2) -- (0,-1pt);
}}}}
\tikzset{Equal/.style={-,double line with arrow={-,-}}}

\hypersetup{
	colorlinks = true, % links instead of boxes
	urlcolor   = blue, % external hyperlinks
	linkcolor  = blue, % internal links
	citecolor  = red   % citations
}

\newcommand{\R}{\mathbb{R}}
\newcommand{\C}{\mathbb{C}}
\newcommand{\Q}{\mathbb{Q}}
\newcommand{\K}{\mathbb{K}}
\newcommand{\N}{\mathbb{N}}
\newcommand{\A}{\mathbb{A}}
\newcommand{\Z}{\mathbb{Z}}
\newcommand{\Hom}{\operatorname{Hom}}
\newcommand{\Ext}{\operatorname{Ext}}
\newcommand{\Tor}{\operatorname{Tor}}

\newcommand{\sfrac}[2]{
    \raisebox{+0.3ex}{$#1$}
    /
    \raisebox{-0.3ex}{$#2$}
}

\newcommand{\ssfrac}[2]{
    \raisebox{+0.3ex}{\scalebox{0.9}{$#1$}}
    /
    \raisebox{-0.3ex}{\scalebox{0.9}{$#2$}}
}

\newcommand\vdashline{
  \tikz[baseline]\draw[thick, dashed](0,-\dp\strutbox)--(0,\ht\strutbox);
}

\renewcommand{\labelitemii}{$\circ$}
\renewcommand{\Im}{\operatorname{Im}}

\newcommand\numberthis{\addtocounter{equation}{1}\tag{\theequation}}

\newtheorem{theorem}{Theorem}[section]
\newtheorem{lemma}{Lemma}[section]

\theoremstyle{definition}
\newtheorem{definition}{Definition}[section]

\theoremstyle{definition}
\newtheorem{example}{Example}[section]

\theoremstyle{remark}
\newtheorem*{remark}{Remark}

\theoremstyle{definition}
\newtheorem{exercise}{Exercise}[section]

\title{Istituzioni di Algebra 2022/2023}
\author{Francesco Minnocci}
\begin{document}
\maketitle
\section*{Homework 3}
\setcounter{section}{3}
\begin{exercise}
	\begin{itemize}\


		\item[(a)]
We consider the injective resolution given by the following exact sequence:
\begin{equation}\label{eq:seq}
\begin{tikzcd}
	0 \rar & {\sfrac{\Z}{p\Z}} \rar["i"] & {\sfrac{\Q}{\Z}} \rar["\pi"] & {\sfrac{\Q}{\Z}} \rar & 0,
\end{tikzcd}
\end{equation}
where the map $i$ is given by multiplication by $\frac{1}{p}$, and $\pi$ is composition of the projection to the cokernel of $i$,
$$\sfrac{\Q}{\Z}\to\sfrac{\sfrac{\Q}{\Z}}{\langle\frac{1}{p}\rangle_{\ssfrac{\Q}{\Z}}},$$ with
the isomorphism \begin{equation}\label{eq:iso}\sfrac{\sfrac{\Q}{\Z}}{\langle\frac{1}{p}\rangle_{\ssfrac{\Q}{\Z}}}\cong\ssfrac{\Q}{\Z}\end{equation} induced by multiplication by $p$ in one direction and division by $p$ in the other one.

Now, applying the $\Hom_{\Z}\left( \ssfrac{\Z}{p\Z},- \right) $ functor to \eqref{eq:seq} yields the left-exact sequence
\begin{equation*}
\begin{tikzcd}
	0 \rightarrow &[-3em] \Hom_{\Z}{\left(\sfrac{\Z}{p\Z},\sfrac{\Z}{p\Z}\right)} \rar["i^\star"] & \Hom_{\Z}{\left(\sfrac{\Z}{p\Z},{\sfrac{\Q}{\Z}}\right)} \rar["\pi^\star"] &
	\Hom_{\Z}{\left(\sfrac{\Z}{p\Z},{\sfrac{\Q}{\Z}}\right)} \rightarrow &[-3em] 0,
\end{tikzcd}
\end{equation*}
where each term is isomorphic to $\ssfrac{\Z}{p\Z}$, which tells us that $$\Ext_{\Z}^i{\left(\sfrac{\Z}{p\Z},\sfrac{\Z}{p\Z}\right)}=0~\forall i>1,$$ and that, since $\pi^\star=0$
by construction,
$$\Ext_{\Z}^1\left( \sfrac{\Z}{p\Z},\sfrac{\Z}{p\Z}
	\right)=\sfrac{\Hom_{\Z}{\left(\sfrac{\Z}{p\Z},{\sfrac{\Q}{\Z}}\right)}}{\Im\left( \pi^\star \right) }\cong\sfrac{\Z}{p\Z}.$$
\item[(b)] Since by \textbf{Proposition} 5.9 any class $$c\in\Ext_{\Z}^1\left( \sfrac{\Z}{p\Z},\sfrac{\Z}{p\Z}\right)=\Hom_{\Z}\left(\sfrac{\Z}{p\Z},\sfrac{\Q}{\Z}\right)$$
	corresponds to an extension class in $\Ext_{\Z}\left( \sfrac{\Z}{p\Z},\sfrac{\Z}{p\Z} \right) $, using a dual construction to that in the proof of the
			aformentioned proposition we can use the pullback of our injective resolution to build the following diagram, which will let us find the desired extension:
\begin{equation*}
\begin{tikzcd}
	0\rar&\sfrac{\Z}{p\Z}\ar[d, Equal]\rar[dashed]&Y\ar[d, dashed]\rar[dashed]&\sfrac{\Z}{p\Z}\ar[d,"c"]\rar&0\\
	0 \rar & {\sfrac{\Z}{p\Z}} \rar["i"] & {\sfrac{\Q}{\Z}} \rar["\pi"] & {\sfrac{\Q}{\Z}} \rar & 0,
\end{tikzcd}
\end{equation*}
where $Y$ is the pullback.

Now, if
\begin{enumerate}
	\item$c=0$: then, we know from the proposition that the corresponding split extension is $$\sfrac{\Z}{p\Z}\oplus\sfrac{\Z}{p\Z};$$ and we can see this from the pullback
		construction as well since $$Y=\left\{ (a,b) \in\sfrac{\Q}{\Z}\oplus\sfrac{\Z}{p\Z}\mid\:\pi(a)=c(b) \right\},$$ and $\pi(a)=0\iff
	a\in\Im(i)\cong\sfrac{\Z}{p\Z}$.
	\item $c\neq 0$ : remembering that we made the identification \eqref{eq:iso}, we have $$c(\overline{1})=\frac{n}{p^2}\in\sfrac{\Q}{\Z};$$
		with $n\neq 0$. In this case the pullback is determined by the condition $$\pi(a)=\frac{nb}{p^2}\iff a=\frac{nb}{p^2}+\frac{m}{p}=\frac{mp+nb}{p^2},$$
		for some $m\in\left\{ 0,\dots,{p-1} \right\} $. This gives us an isomorphism
		$$Y\overset{\Phi}{\cong}\sfrac{\Z}{p^2\Z},$$
		where $$\Phi(a,b)=mp+nb$$ and $$\Phi^{-1}(q)=\left(\frac{q}{p^2},j\cdot n^{-1}\right),$$
		where $q=ip+j\in\frac{\Z}{p^2\Z}$.
\end{enumerate}
\end{itemize}
\end{exercise}
\begin{exercise}
First, notice that by the structure theorem for finitely generated abelian groups we have
$$ A\cong A_\text{tors}\oplus\Z^k ,$$
for some $k\in \N$.

Now, take the following injective resolution of $\Z$:
\begin{equation*}
\begin{tikzcd}
	0 \rar & {\Z} \rar["i"] & {\Q} \rar["\pi"] & {\sfrac{\Q}{\Z}} \rar & 0,
\end{tikzcd}
\end{equation*}
Then, applying the $\Hom_{\Z}\left( A,- \right) $ functor yields the left-exact sequence
\begin{equation}\label{eq:hom}
\begin{tikzcd}
	0 \rightarrow &[-3em] \Hom_{\Z}{\left(A,\Z\right)} \rar["i^\star"] & \Hom_{\Z}{\left(A,\Q\right)} \rar["\pi^\star"] &
	\Hom_{\Z}{\left(A,{\sfrac{\Q}{\Z}}\right)} \rightarrow &[-3em] 0,
\end{tikzcd}
\end{equation}
and we notice that, since for any $\Z$-module $B$ $$\Hom_{\Z}\left( A,B \right) \cong\Hom_{\Z}\left( A_\text{tors},B \right)\oplus\Hom_{\Z}\left( \Z^k,B \right).$$

Also, if $B=\Z$ or $B=\Q$ then  $\Hom_{\Z}\left( A_\text{tors},B \right)=0,$ as each element of $A_\text{tors}$ has finite order, must go to an element of finite order
in $B$, which is necessarily $0$).

Thus, can reduce the sequence in \eqref{eq:hom} through successive isomorphisms to:
\begin{equation*}
\begin{tikzcd}
	0 \rar & \Hom_{\Z}{\left(\Z^k,\Z\right)} \rar["i^\star"]\ar[d,"\cong"]&\Hom_{\Z}{\left(\Z^k,\Q\right)} \rar["\pi^\star"]\ar[d,"\cong"] &
	\Hom_{\Z}{\left(A,{\sfrac{\Q}{\Z}}\right)} \rar\ar[d,"\cong"] & 0\\
	0 \rar & {\Z^k} \rar["i^\star"] & {\Q^k} \rar["\pi^\star"] & {\left(\sfrac{\Q}{\Z}\right)^k\oplus\Hom_{\Z}\left( A_\text{tors},\sfrac{\Q}{\Z} \right)} \rar & 0.
\end{tikzcd}
\end{equation*}
Here, the induced map $\pi^\star$ is the projection of $\Q^k$ to $\left(\ssfrac{\Q}{\Z}\right)^k$; therefore, we are ready to compute the first cohomology group of this sequence:
$$\Ext_{\Z}^1\left( A,\Z \right)=\frac{\left(\sfrac{\Q}{\Z}\right)^k\oplus\Hom_{\Z}\left( A_\text{tors},\sfrac{\Q}{\Z} \right)}{\Im\left( \pi^\star \right) }\cong\Hom_{\Z}\left( A_\text{tors},\sfrac{\Q}{\Z} \right).$$
\end{exercise}
\begin{exercise}
	\begin{itemize}\


		\item[(a)] Since $A$ is and local with maximal ideal $(\overline{t})$, and since (\textbf{Proposition} 2.17) projective modules over local rings are free, once we notice that
	$$ \overline{t}\cdot(\overline{t})=0 ,$$
	we get that $\ker\left( p \right) $ cannot be free as an $A$-module, and thus cannot be projective as well.
		\item[(b)] We observe that we can extend the exact sequence
			$$ 0\rightarrow \ker{(p)}\overset{i}{\hookrightarrow} A\overset{p}{\twoheadrightarrow}\K\rightarrow 0 $$
			to a resolution of $\K$ like so:
			\begin{equation}\label{eq:res}
			\begin{tikzcd}[column sep=small]
				\cdots\rar&A\rar["\cdot t"]&{A} \rar["\cdot t"]\ar[d, "\pi", two heads] & {A} \rar["p", two heads] & {\K} \rar & 0.\\
				&&{\ker{(p)}} \ar[ru, "i", hook]
			\end{tikzcd}
			\end{equation}
			This way, we have an infinite projective (as $A$ is free on itself and thus projective) resolution of $\K$ as an $A$-module, since $$\ker{(p)}=t\cdot A$$ and
			$$\Im{(\cdot t)}=t\cdot A.$$
		\item[(c)] We apply the $-\otimes\K$ functor to the resolution \eqref{eq:res}:
			\begin{equation*}
			\begin{tikzcd}
				\cdots\rar&{A\otimes\K}\rar["\cdot t\otimes\text{id}_{\K}"]&{A\otimes\K} \rar["p\otimes\text{id}_{\K}"] & {\K\otimes\K}.
			\end{tikzcd}
			\end{equation*}
			Now, each term in the above is isomorphic to $\K$ by known properties of the tensor product; moreover, the induced maps are all $0$. This is obvious for the
		ones coming from multiplication by $t$, and we verify it directly for the last map:
		$$ (p\otimes\text{id}_{\K})(a\otimes k)=ta\otimes k=a\otimes tk=0~\forall a\in A,\:k\in\K .$$
		Therefore, we have that $$\Tor^d_A{(\K,\K)}\cong\K~\forall d,$$
		and thus by the characterization of global dimension given in \textbf{Corollary 6.9} we have that $$\operatorname{gldim}(A)=\operatorname{pd}(\K)=+\infty.$$
	\end{itemize}

\end{exercise}
\begin{exercise}
(done in collaboration with Lucio Tanzini)\

	\begin{itemize}
		\item[(a)] Since $M$ is finitely generated, there exists an $r$ such that $A^{r}$ projects onto $M$. Now, the kernel of such map is finitely generated (as $A$ is
		Noetherian), so we can reiterate such construction by defining:
		\begin{align*}
			M^0&:=M\\
			M^i&:=\ker\left( A^{r_{i-1}}\twoheadrightarrow M^{i-1} \right),~i\geq1\\
			r_i&:=\operatorname{rk}\left( M^i \right),~i\geq0.
		\end{align*}
		Then, we have the following exact sequence:
		\begin{equation*}
			\begin{tikzcd}[column sep=small]
				&&&&{M^1} \ar[rd, "i", hook]\\
				0\rar & M^d\rar&{A^{r_{d-1}}}\rar&{\cdots}\rar&{A^{r_1}} \rar\ar[u, two heads] & {A^{r_0}} \rar[two heads] & {M} \rar & 0.
			\end{tikzcd}
		\end{equation*}
	\item[(b)] Notice that we can replace $A$ with $\K[x_1,\dots,x_d]_{(x_1,\dots,x_d)}$, as $M^d$ is projective if and only if $M^d\otimes\K$ is, since being projective is a
		local property.

		Then, we: prove inductively that $(x_1,\dots,x_i)$ is a regular sequence in $M^i$ for all $i\leq d$, or equivalently that multiplication by any $x_i$ is injective
		in $\ssfrac{M^i}{(x_1,\dots,x_i)}$.

		If $i=1$, $M^1\hookrightarrow A^{r_0}$, thus multiplication by $x_1$ is injective on $M^1$ as it injects in a direct sum of copies of $A$ and $x_1$ is not a
		zero-divisor in $A$.

		If $i>1$, the inductive hypothesis tells us that $(x_1,\dots,x_{i-1})$ is regular on $M^{i-1}$; and we show by induction that, for $0\leq k\leq
		i-2$ (using the
		notation $X_k:=(x_1,\dots,x_k)$), multiplication by $x_{k+1}$ is injective and \begin{equation}\label{eq:snake}
			0\rightarrow \sfrac{M^i}{X_k M^i}\rightarrow
			\sfrac{A^{r_{i-1}}}{X_k
		A^{r_{i-1}}}\rightarrow\sfrac{M^{i-1}}{X_k M^{i-1}}\rightarrow 0
	\end{equation}
		is exact.

		If $k=0$, then this is true by construction of the $M^i$s; and if $k>0$ we can apply the Snake Lemma to the sequence in \eqref{eq:snake} (using $X_{k-1}$) on both
		rows (which is exact by inductive hypothesis), and with vertical
		maps multiplication by $x_k$, which are all three injective by inductive hypothesis, to obtain the exact sequence of cokernels
		$$ 0\rightarrow \sfrac{M^i}{X_k M^i}\rightarrow \sfrac{A^{r_{i-1}}}{X_k A^{r_{i-1}}}\rightarrow\sfrac{M^{i-1}}{X_k M^{i-1}}\rightarrow 0 ,$$
		as desired. Furthermore, since this means that $\ssfrac{M^i}{X_k M^i}$ injects into a direct sum of copies of $\ssfrac{A}{X_k}$, by the same similar reasoning as in
		the case $i=1$ multiplication by $x_{k+1}$ is injective in $\ssfrac{A}{X_k}$, and thus in $\ssfrac{M^i}{X_k M^i}$.

		Now, we show by reverse induction on $k\leq d$ that $\ssfrac{M^d}{X_k M^d}$ is free over $\ssfrac{A}{X_k}$: if $k=d$, then $\ssfrac{M^d}{X_k M^d}$ is a vector space
		over $\K$, and thus it is free. Otherwise, suppose that the stament holds for $k+1$, let $t$ be the rank of $\ssfrac{M^d}{X_{k+1} M^d}$ over $\ssfrac{A}{X_{k+1}}$,
		and consider the following diagram:
		\begin{equation*}
			\begin{tikzcd}[column sep=small]
				\left(\ssfrac{A}{X_{k}}\right)^t\ar[d, dashed, "\overline{\varphi}"]\rar[two heads]\ar[rd,"\varphi"]&\left(\ssfrac{A}{X_{k+1}}\right)^t\ar[d,"\cong"]\\
				\ssfrac{M^d}{X_{k} M^d}\rar[two heads]&\ssfrac{M^d}{X_{k+1} M^d}.
			\end{tikzcd}
		\end{equation*}
		Here, $\varphi$ is given by composition, and we construct $\overline{\varphi}$ by lifting images of $\varphi$, so that it makes the diagram commute. To show that it is surjective, we notice that, if
		$N$ is the submodule of $\sfrac{M^d}{X_k M^d}$ generated by the image of $\overline{\varphi}$, and since the horizontal maps are projections with respect to the
		image of multiplication by $x_{k+1}$, we have
		$$ \sfrac{M^d}{X_k M^d}=N+x_{k+1}\cdot\sfrac{M^d}{X_k M^d} ,$$
		and since $x_{k+1}$ is contained in the maximal ideal of $\ssfrac{A}{X_k}$, we can conclude using \textbf{Nakayama} that
		$$ N=\Im{\varphi}=\sfrac{M^d}{X_k M^d} .$$
		We therefore only need injectivity of $\overline{\varphi}$, which follows from the injectivity of multiplication by $x_{k+1}$ we proved previously: take the diagram

		\begin{equation*}
			\begin{tikzcd}
				0 \rar & \left(\ssfrac{A}{X_{k}}\right)^t \rar["\cdot x_{k+1}", hook,]\ar[d, "\overline{\varphi}"]&\left(\ssfrac{A}{X_{k}}\right)^t \rar[two heads]\ar[d,"\overline{\varphi}"] & \left(\ssfrac{A}{X_{k+1}}\right)^t \rar\ar[d,"\cong"] & 0\\
			0 \rar & \sfrac{M^d}{X_k M^d} \rar["\cdot x_{k+1}", hook] & \sfrac{M^d}{X_k M^d} \rar[two heads] & \sfrac{M^d}{X_{k+1} M^d} \rar & 0.
			\end{tikzcd}
		\end{equation*}
		Since we have shown that  multiplication by $x_{k+1}$ is injective, both rows are exact and we can apply the Snake Lemma to obtain an isomorphism between
		$\ker{(\overline{\varphi})}$ and $x_{k+1}\cdot\ker{(\overline{\varphi})}$, which means that we can apply \textbf{Nakayama} to get $\ker{(\overline{\varphi})}=0$,
		which yields an isomorphism between $\ssfrac{M^d}{X_k M^d}$ and a direct sum of copies of $\left( \ssfrac{A}{X_k} \right)$, as desired.

	In conclusion, for $k=0$ we get that $M^d$ is a free module of rank $t$ over $A$, and is thus projective.

	Finally, by \textbf{Proposition} 6.2 we can conclude that $\text{pd}(M)\leq d$, which gives us Hilbert's Syzygy Theorem.
	\end{itemize}

\end{exercise}
\begin{exercise}
	Let $(x_1,\dots,x_d)$ be a regular sequence in $A$. Then, by \textbf{Corollary} 8.14, we have $$\Ext_A^i\left( \K,\K \right)\cong H^i\left( \Hom\left( K(\underline{x}),\K \right)  \right),$$
	and since $$ K(\underline{x})_i=\Lambda^i{A^d}\cong A^{\binom{d}{i}} ,$$
	where the isomorphism follows from \textit{Facts} 7.11 (6), we have that $$\Hom\left( \Lambda^i A^d,\K \right)\cong\K^{\binom{d}{i}}.$$
	Therefore, we only need to show that the cohomology of the above complex picks out each term, which is true because the differentials are all zero: indeed, if $i>0$ take
	$\phi\in\Hom\left( \Lambda^i A^d,\K \right)$. Then, we have
	$$d_{\underline{x}}^i{(\phi)}\left( \bigwedge_{k=1}^{i+1}{\underline{a}_{k}} \right) =\phi\circ
	d_{\underline{x}}{\left(\bigwedge_{k=1}^{i+1}{\underline{a}_{k}}\right)}=\phi\left( \sum_{k=1}^{i+1}{(-1)^{k+1}\left( b_1 x_1+\dots+b_d x_d \right)\cdot
	\bigwedge_{k=1,k\neq i}^{i+1}{\underline{a}_k} } \right)=0,$$
	where $\underline{b}$ is the scalar product of $\underline{a}$ and $\underline{x}$, and the last equality follows from linearity of $\phi$ and the fact that the
	$x_1,\dots,x_d$ map to 0 in $\K$. The case $i=0$ is the same if we understand the empty wedge product to be $1$.
\end{exercise}
\end{document}
